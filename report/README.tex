\documentclass[11pt,a4paper]{article}

\usepackage[margin=1in]{geometry}
\usepackage{indentfirst}
\usepackage{listings}
\usepackage{color}
\usepackage{hyperref}
\usepackage[utf8x]{inputenc}

\definecolor{dkgreen}{rgb}{0,0.3,0}
\definecolor{gray}{rgb}{0.5,0.5,0.5}
\definecolor{mauve}{rgb}{0.58,0,0.82}
\definecolor{orange}{rgb}{.7,.3,0}

\makeatletter
\lst@InstallKeywords k{types}{typestyle}\slshape{typestyle}{}ld
\lst@InstallKeywords k{prompt}{promptstyle}\slshape{promptstyle}{}ld
\makeatother

\lstset{
    language=Haskell,
    aboveskip=3mm,
    belowskip=3mm,
    showstringspaces=false,
    upquote=true,
    basicstyle={\small\ttfamily},
    keywordstyle=\color{blue},
    commentstyle=\color{gray},
    stringstyle=\color{mauve},
    typestyle=\color{orange},
    promptstyle=\color{dkgreen},
    escapeinside={\%@}{@)},
    breaklines=true,
    breakatwhitespace=true,
    morestring=[s]{'}{'},
    moretypes={
        Map, Natural, Variable, Exponent, Coefficient, Monomial, Polynomial, Differentiable
    },
    moreprompt={
        ghci>
    },
    alsoletter=>
}

\title{PFL Project}
\author{João Pereira, Nuno Pereira}

\begin{document}

\maketitle

\pagebreak

\begin{abstract}
    TODO
\end{abstract}

\tableofcontents

\pagebreak

\section{Introduction}

The goal of this project was to implement polynomials and common operations performed on them, such as addition or derivation, using the Haskell programming language.

\section{Internal Representation}

For the internal representation of the polynomial data structure, we implemented the following:

\begin{lstlisting}
data Natural = One | Suc Natural
    deriving (Eq, Ord)

type Variable = Char

type Exponent = Natural

type Coefficient = Double

data Monomial = Monomial Coefficient (Map Variable Exponent)
    deriving (Eq)

newtype Polynomial = Polynomial [Monomial]
    deriving (Eq)
\end{lstlisting}

This allows us to represent \lstinline{Polynomial}s and \lstinline{Monomial}s in a way that naturally represents what they are, while also ensuring that the operations performed on them are efficient:

\begin{itemize}
    \item doing work on a \lstinline{Polynomial} is (almost) the same as doing the same work to each of its \lstinline{Monomial}s;

    \item working with the variables and degrees of each \lstinline{Monomial} is not only simplified but also more efficient because of the nature of the underlying \lstinline{Map} data structure.

          For example, normalizing a polynomial is done in $ \mathcal{O} \left( k' k m \times \log \left( \frac{n+1}{m+1} + k' \right) \right), m \leq n $ time instead of $ \mathcal{O} \left( m k^2 \right) $, like we had in a previous implementation:
          % these complexities are getting very strange but the comparison term is the naive "nub-like" solutions which has a worst-case complexity of O(mk²), where k is the number of monomial and m is the number of exponents
          % need to check if it actually is more efficient
          \begin{itemize}
              \item $ \mathcal{O} \left( m \times \log \left( \frac{n+1}{m+1} \right) \right), m \leq n $ for aggregating any 2 \lstinline{Monomial}s, where $n$ and $m$ are the sizes of the \lstinline{Monomial}s' "exponent map";

              \item $ \mathcal{O} (k-1) = \mathcal{O} (k) $, where $k$ is the number of \lstinline{Monomial}s in the original \lstinline{Polynomial};

              \item $ \mathcal{O} (k' \times \log(k')) $ for sorting the aggregated \lstinline{Monomial}s, where \textit{k'} is the number of \lstinline{Monomial}s in the \lstinline{Polynomial} that resulted from the previous step;
          \end{itemize}

    \item ensuring that the exponents used are \lstinline{Natural} numbers better models the mathematical definition of a \lstinline{Monomial} as well as allowing to catch unexpected bugs arising from the use of negative exponents.

\end{itemize}

\pagebreak

\section{Implementation}

\subsection{Polynomial}

\begin{itemize}
    \item \begin{lstlisting}
normalize :: Polynomial -> Polynomial
normalize (Polynomial p) = Polynomial $ sortOn Down [Monomial c exps | (exps, c) <- toList (normalizeHelper p), c /= 0]
    where
        normalizeHelper :: [Monomial] -> Map (Map Variable Exponent) Coefficient
        normalizeHelper [] = empty
        normalizeHelper ((Monomial c exps) : xs) = unionWith (+) (fromList [(exps, c)]) (normalizeHelper xs)    
    \end{lstlisting}

          In the \lstinline{normalize} function we process each \lstinline{Monomial} that composes the \lstinline{Polynomial} only once, "accumulating" the desired results, which are then re-ordered, converted back into \lstinline{Monomial}s and sorted. This strategy works because, by mapping an "exponent map" to a coefficient and exploiting the \lstinline{unionWith (+)} function, we only need to calculate the union of all the \lstinline{Monomial}s: \lstinline{Monomial}s with the same "exponent map" simply have their coefficients added.

    \item \begin{lstlisting}
instance Show Polynomial where
    show :: Polynomial -> String
    show (Polynomial []) = ""
    show (Polynomial [m]) = show m
    show (Polynomial (m : (Monomial c e) : ms)) = show m ++ showSign c ++ show (Polynomial (abs (Monomial c e) : ms))
        where
            showSign c = if c < 0 then " - " else " + "
    \end{lstlisting}

          In the \lstinline{show} function the \lstinline{Monomial}s are printed one at a time using the coefficient of the next \lstinline{Monomial} in the list to define the sign to show: '\lstinline{-}' if it is negative, '\lstinline{+}' otherwise.

    \item \begin{lstlisting}
instance Differentiable Polynomial where
    (//) :: Polynomial -> Variable -> Polynomial
    p // v = normalize $ p !// v
    \end{lstlisting}

          Makes a \lstinline{Polynomial} inherently differentiable. This functions basically differentiates each \lstinline{Monomial} and then normalizes the output.

    \item \begin{lstlisting}
instance Num Polynomial where
    (+) :: Polynomial -> Polynomial -> Polynomial
    x + y = normalize $ x !+ y

    (*) :: Polynomial -> Polynomial -> Polynomial
    x * y = normalize $ x !* y
    
    abs :: Polynomial -> Polynomial
    abs (Polynomial x) = Polynomial $ map abs x
    
    signum :: Polynomial -> Polynomial
    signum = error "Not implemented"
    
    fromInteger :: Integer -> Polynomial
    fromInteger x = Polynomial [fromInteger x]
    
    negate :: Polynomial -> Polynomial
    negate (Polynomial x) = Polynomial $ map negate x
    \end{lstlisting}

          Makes a \lstinline{Polynomial} be treated as a \lstinline{Num} so that common number operations can be performed on \lstinline{Polynomial}s. This is basically a way to perform the same operations on all the \lstinline{Monomial}s that make up a \lstinline{Polynomial}.

    \item \begin{lstlisting}
instance Read Polynomial where
    readsPrec :: Int -> ReadS Polynomial
    readsPrec _ s = [(Polynomial $ read s, "")]
    \end{lstlisting}

          Utility instantiation of \lstinline{Read} so that \lstinline{Polynomial}s can be parsed directly from an input string.

    \item \begin{lstlisting}
(!+) :: Polynomial -> Polynomial -> Polynomial
Polynomial x !+ Polynomial y = Polynomial $ x ++ y
    \end{lstlisting}

          Adds two \lstinline{Polynomial}s, without normalizing the output. Since \lstinline{Polynomial}s are just a list of \lstinline{Monomial}s, this just creates a \lstinline{Polynomial} that results from the concatenation of all the \lstinline{Monomial}s in the two input \lstinline{Polynomial}s.

    \item \begin{lstlisting}
(!-) :: Polynomial -> Polynomial -> Polynomial
x !- y = x !+ negate y
    \end{lstlisting}

          Subtracts two \lstinline{Polynomial}s, without normalizing the output. This just calls \lstinline{(!+)} but having the second \lstinline{Polynomial} negated.

    \item \begin{lstlisting}
(!*) :: Polynomial -> Polynomial -> Polynomial
Polynomial x !* Polynomial y = Polynomial [i * j | i <- x, j <- y]
    \end{lstlisting}

          Multiplies both \lstinline{Polynomial}s, without normalizing the output. This is basically performing a Cartesian product between the inputs' \lstinline{Monomial}s.

    \item \begin{lstlisting}
(!//) :: Polynomial -> Variable -> Polynomial
Polynomial p !// v = Polynomial $ map (// v) p
    \end{lstlisting}

          Differentiates a \lstinline{Polynomial} without normalizing the output. This just applies a mapping to differentiate each of this \lstinline{Polynomial}'s \lstinline{Monomial}s.

\end{itemize}

\subsection{Monomial}

\begin{itemize}
    \item \begin{lstlisting}
instance Differentiable Monomial where
    (//) :: Monomial -> Variable -> Monomial
    Monomial n m // v = case Data.Map.lookup v m of
        Nothing -> 0
        Just One -> Monomial n (delete v m)
        Just exp -> Monomial (n * fromInteger (toInteger exp)) (adjust (\x -> x - One) v m)      
    \end{lstlisting}

          Makes a \lstinline{Monomial} inherently differentiable. If the new exponent of the variable of differentiation in the \lstinline{Monomial} is 1, then the exponent is removed from this \lstinline{Monomial}'s "exponent map". Otherwise, the model remains the same, having in mind the differentiation rules of mathematical monomials.

    \item \begin{lstlisting}
instance Num Monomial where
    (+) :: Monomial -> Monomial -> Monomial
    Monomial c1 m1 + Monomial c2 m2
        | m1 == m2 = Monomial (c1 + c2) m1
        | otherwise = error "Can't add monomials with different degrees"
    
    (*) :: Monomial -> Monomial -> Monomial
    Monomial c1 m1 * Monomial c2 m2 = Monomial (c1 * c2) (unionWith (+) m1 m2)
    
    abs :: Monomial -> Monomial
    abs (Monomial n m) = Monomial (abs n) m
    
    signum :: Monomial -> Monomial
    signum (Monomial n m) = Monomial (signum n) empty
    
    fromInteger :: Integer -> Monomial
    fromInteger n = Monomial (fromInteger n) empty
    
    negate :: Monomial -> Monomial
    negate (Monomial n m) = Monomial (-n) m
    \end{lstlisting}

          Makes a \lstinline{Monomial} be treated as a \lstinline{Num}, so that common number operations can be performed on them.

    \item \begin{lstlisting}
instance Ord Monomial where
    (<=) :: Monomial -> Monomial -> Bool
    Monomial ca ea <= Monomial cb eb
        | ea == eb = ca <= cb
        | otherwise = or $ zipWith compare (toList ea) (toList eb)
        where
            compare (v1, e1) (v2, e2)
                | v1 == v2 = e1 <= e2
                | otherwise = v1 >= v2
    \end{lstlisting}

          Makes it so that \lstinline{Monomial}s can be ordered. This is especially useful when normalizing \lstinline{Polynomial}s.

    \item \begin{lstlisting}
instance Show Monomial where
    show :: Monomial -> String
    show (Monomial c m)
        | Data.Map.null m = showReal c
        | c == 1 = showVars m
        | c == (-1) = '-' : showVars m
        | otherwise = showReal c ++ showVars m
        where
            showVars m = concat $ mapWithKey showVar m
            showVar v 1 = [v]
            showVar v e = v : showSuperscript (toInteger e)
    \end{lstlisting}

          Utility instanciation of \lstinline{Show} for easier printing of \lstinline{Monomial}s.

    \item \begin{lstlisting}
instance Read Monomial where
    readsPrec :: Int -> ReadS Monomial
    readsPrec _ s = [readMonomial s]
        where
        readMonomial :: String -> (Monomial, String)
        readMonomial s' = (Monomial coeff vars, s''')
            where
            (coeff, s'') = readCoefficient s'
            (vars, s''') = readVars s''
    
        readCoefficient :: String -> (Coefficient, String)
        readCoefficient ('-' : cs) = (-f, s)
            where
            (f, s) = readCoefficient cs
        readCoefficient ('+' : cs) = readCoefficient cs
        readCoefficient (' ' : cs) = readCoefficient cs
        readCoefficient cs = case f of
            "" -> (1, s)
            f -> (read f, s)
            where
            (f, s) = span (\c -> isDigit c || c == '.') cs
    
        readVars :: String -> (Map Variable Exponent, String)
        readVars vs = (fromList $ readVarsHelper f, s)
            where
            (f, s) = span (\c -> isSpace c || isDigit c || isAsciiLower c || c == '*' || c == '^') vs
    
            readVarsHelper :: String -> [(Variable, Exponent)]
            readVarsHelper "" = []
            readVarsHelper (' ' : s) = readVarsHelper s
            readVarsHelper s = (v, e) : readVarsHelper s''
                where
                readVar :: String -> (Variable, String)
                readVar "" = error "No read"
                readVar (v : vs)
                    | v == '*' || v == ' ' = readVar vs
                    | isAsciiLower v = (v, vs)
                    | otherwise = error "No read"
    
                (v, s') = readVar s
    
                readExponent :: String -> Exponent
                readExponent "" = One
                readExponent ('^' : es) = readExponent es
                readExponent (' ' : es) = readExponent es
                readExponent e = fromInteger $ read e
    
                (e, s'') = (readExponent es, s'')
                    where
                    (es, s'') = span (\c -> isSpace c || isDigit c || c == '^') s'
    
    readList :: ReadS [Monomial]
    readList s = [(helper s, "")]
        where
        helper "" = []
        helper s = m : helper s'
            where
            (m, s') = head (reads s :: [(Monomial, String)])
    \end{lstlisting}

          Utility instanciation of \lstinline{Read} that allows a \lstinline{Monomial} to be parsed an input string.

\end{itemize}

\subsection{Natural}

\begin{itemize}
    \item \begin{lstlisting}
instance Enum Natural where
    toEnum :: Int -> Natural
    toEnum n
        | n == 1 = One
        | n > 1 = Suc $ toEnum (n - 1)
        | otherwise = error "Cannot be negative"
    
    fromEnum :: Natural -> Int
    fromEnum One = 1
    fromEnum (Suc n) = 1 + fromEnum n
    \end{lstlisting}

          Utility instantiation of \lstinline{Enum} that allows \lstinline{Natural}s to be convert from and to \lstinline{Integer}s.

    \item \begin{lstlisting}
instance Num Natural where
    (+) :: Natural -> Natural -> Natural
    a + b = toEnum $ fromEnum a + fromEnum b
    
    (*) :: Natural -> Natural -> Natural
    a * b = toEnum $ fromEnum a * fromEnum b
    
    (-) :: Natural -> Natural -> Natural
    a - b
        | number <= 0 = error "Negative difference"
        | otherwise = toEnum number
        where
        number = fromEnum a - fromEnum b
    
    abs :: Natural -> Natural
    abs n = n
    
    signum :: Natural -> Natural
    signum p = 1
    
    fromInteger :: Integer -> Natural
    fromInteger n
        | n < 1 = error "Must be positive"
        | n == 1 = One
        | otherwise = Suc (fromInteger (n Prelude.- 1))
    
    negate :: Natural -> Natural
    negate = error "Cannot negate value"
    \end{lstlisting}

          Makes it so that \lstinline{Natural}s are treated as native number types.

    \item \begin{lstlisting}
instance Show Natural where
  show :: Natural -> String
  show = show . fromEnum
    \end{lstlisting}

          Utility instantiation of \lstinline{Show} that facilitates printing \lstinline{Natural} numbers.

    \item \begin{lstlisting}
instance Real Natural where
    toRational :: Natural -> Rational
    toRational x = toInteger x % 1
    \end{lstlisting}

          Utility instantiation of \lstinline{Real} that is needed to be able to instantiate \lstinline{Integral}.

    \item \begin{lstlisting}
instance Integral Natural where
    quotRem :: Natural -> Natural -> (Natural, Natural)
    quotRem x y = (toEnum q, toEnum r)
        where
        (q, r) = quotRem (fromEnum x) (fromEnum y)
    
    toInteger :: Natural -> Integer
    toInteger = toInteger . fromEnum      
    \end{lstlisting}

          Utility instantiation of \lstinline{Integral} that makes it so that \lstinline{Natural} numbers are treated as \lstinline{Integer} numbers.

\end{itemize}

\section{Usage examples}

As the program is designed to be used inside of \lstinline{ghci}, all examples will assume the program has been ran as \lstinline{ghci Main}.

\subsection{Inputting and outputting polynomials}

A polynomial can be inputted directly as its internal representation:

\begin{lstlisting}
ghci> Polynomial [Monomial 0 (fromList [('x', 2)]), Monomial 2 (fromList [('y', 1)]), Monomial 5 (fromList [('z', 1)]), Monomial 1 (fromList [('y', 1)]), Monomial 7 (fromList [('y', 2)])]
%@0x² + 2y + 5z + y + 7y²@)
\end{lstlisting}

Or by using \lstinline{read} and inputting a properly formatted\footnote{The format for a monomial is \lstinline{[-+]?\\d*\\.\\d*(\\*?[a-z]\\^?\\d*)*}} string:

\begin{lstlisting}
ghci> read "0*x^2 + 2*y + 5*z + y + 7*y^2" :: Polynomial
%@0x² + 2y + 5z + y + 7y²@)
\end{lstlisting}

A simplified format is also supported:

\begin{lstlisting}
ghci> read "0x2 + 2y + 5z + y + 7y2" :: Polynomial
%@0x² + 2y + 5z + y + 7y²@)
\end{lstlisting}

Polynomials will always be outputted in their formatted form by \lstinline{ghci}, as the class \lstinline{Show} has been instanced.
As so, \lstinline{show} can also be used to get a string with the formatted polynomial.

\begin{lstlisting}
ghci> show (read "0x2 + 2y + 5z + y + 7y2" :: Polynomial)
"0x\178 + 2y + 5z + y + 7y\178"
\end{lstlisting}

\subsection{Normalizing polynomials}

A polynomial can be normalized using the \lstinline{normalize} function.
Polynomials will also get normalized after most operations, but this can be skipped.

\begin{lstlisting}
ghci> normalize $ read "0*x^2 + 2*y + 5*z + y + 7*y^2" :: Polynomial
%@7y² + 3y + 5z@)
\end{lstlisting}

\subsection{Adding, subtracting and multiplying polynomials}

The following examples will use the polynomials:

$$ P_1 = x^3 + x^2 + 12y + 0 $$
$$ P_2 = 4x + 5y + 8 + 10z $$

To add polynomials, use the \lstinline{(+)} operator, or the \lstinline{(!+)} operator, if normalization is to be skipped.

\begin{lstlisting}
ghci> p1 + p2
%@x³ + x² + 4x + 17y + 10z + 8@)
ghci> p1 !+ p2
%@x³ + x² + 12y + 0 + 4x + 5y + 8 + 10z@)
\end{lstlisting}

To subtract polynomials, use the \lstinline{(-)} operator, or the \lstinline{(!-)} operator, if normalization is to be skipped.

\begin{lstlisting}
ghci> p1 - p2
%@x³ + x² - 4x + 7y - 10z - 8@)
ghci> p1 !- p2
%@x³ + x² + 12y + 0 - 4x - 5y - 8 - 10z@)
\end{lstlisting}

To multiply polynomials, use the \lstinline{(*)} operator, or the \lstinline{(!*)} operator, if normalization is to be skipped.

\begin{lstlisting}
ghci> p1 * p2
%@4x⁴ + 12x³ + 5x³y + 10x³z + 8x² + 48xy + 5x²y + 10x²z + 60y² + 96y + 120yz@)
ghci> p1 !* p2
%@4x⁴ + 5x³y + 8x³ + 10x³z + 4x³ + 5x²y + 8x² + 10x²z + 48xy + 60y² + 96y + 120yz + 0x + 0y + 0 + 0z@)
\end{lstlisting}

\subsection{Differentiating polynomials}

To differentiate a polynomial, use the \lstinline{(//)} operator, or the \lstinline{(!//)} operator, if normalization is to be skipped.
The second argument is the variable to differentiate by.

\begin{lstlisting}
ghci> (read "x3 + x2 + 12y + 5" :: Polynomial) // 'x'
%@3x² + 2x@)
ghci> (read "x3 + x2 + 12y + 5" :: Polynomial) !// 'x'
%@3x² + 2x + 0 + 0@)
ghci> (read "x3 + x2 + 12y + 5" :: Polynomial) // 'y'
%@12@)
ghci> (read "x3 + x2 + 12y + 5" :: Polynomial) !// 'y'
%@0 + 0 + 12 + 0@)
\end{lstlisting}

\subsection{Other operations}

To get the absolute value of a polynomial, use the \lstinline{abs} function.
This will make all coefficients non-negative.

\begin{lstlisting}
ghci> abs $ read "-x3 + x2 - 12y + 5" :: Polynomial
%@x³ + x² + 12y + 5@)
\end{lstlisting}

To negate a polynomial, use the \lstinline{negate} function.
This will flip the sign of all coefficients.

\begin{lstlisting}
ghci> negate $ read "-x3 + x2 - 12y + 5" :: Polynomial
%@x³ - x² + 12y - 5@)
\end{lstlisting}

\pagebreak

\appendix
\section{Appendix}

\subsection{Figures}

TODO

\end{document}
